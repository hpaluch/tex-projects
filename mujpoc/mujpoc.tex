% encoding UTF-8
% Můj počítač 0.0
%
%
%
\chyph  % requires csplain (pdf)TeX format
%\input czech.sty
%\bigskipamount=12pt
%\medskipamount=6pt
%\smallskipamount=3pt
\raggedbottom   % v nouzi může vynechat místo dole na stránce
%%%%%%%%%%%%%%%%%%%%%%%%%
%%%%%%%% Fonty
%%%%%%%%%%%%%%%%%%%%%%%%%

\font\ftitul=csssbx10 scaled \magstep5
\font\fautor=csssdc10 scaled \magstep3
\font\bbf=csbx10 scaled \magstep4
\font\bmf=csbx10 scaled \magstep2
\font\fsec=csbx10 scaled \magstep1
\font\fuvnap=csssqi8 scaled \magstep4
\font\fuvtext=csti10
\font\sc=cscsc10
\font\consym=conn
\font\eightrm=csr8

%%%%%%%%%%%%%%%%%%%%%%%%%
%%%%%%%%% Makra
%%%%%%%%%%%%%%%%%%%%%%%%%

% Registry
\newcount\kapnum	% číslování kapitol
\newcount\secnum	% číslování odstavců (sekcí)
\newread\test
\newwrite\toc

% \newpage - nová stránka
\def\newpage{\par\vfill\penalty-10000\break}


% \softinput - test na přítomnost souboru (pro obsah, atd.)
\def\softinput#1{\immediate\openin\test=#1
	\ifeof\test\message{Varování: soubor #1 nebyl nalezen. Spustťě ještě jednou}%
	  \else\immediate\closein\test\input #1\fi}

%% generovaní obsahu
% \heretoc - vysází obsah na určené místo
% \maketoc - tvorba referencí pro obsah

\def\writetoc#1#2{}
\def\maketoc{\immediate\openout\toc=\jobname.toc
       \def\writetoc##1##2{\write\toc{\string\tocline{\string##1}%
                {##2}{\the\pageno}}}}
\output={\lineskiplimit=0pt \plainoutput} % výstupní rutina stránky
\def\heretoc{\baselineskip=16pt \lineskiplimit=-\maxdimen
      \leftline{\bbf \vrule height 18pt width 0pt Obsah}\nobreak\vskip 8pt
      \softinput{\jobname.toc}\par \kapnum=0}
\def\tocline#1{#1}
\def\rdotfill{\leaders\hbox to 5pt{\hss.\hss}\hfill}
\def\tkap#1#2{\advance\kapnum by 1
	      \secnum=0
	      \line{\hbox to 1.5em{\hfil\bf\the\kapnum. }
	      #1\unskip\rdotfill\ #2}}
\def\tsec#1#2{\advance\secnum by 1
              \line{\hbox to 40pt{\hfil\bf\the\kapnum.\the\secnum\ }
               #1\unskip\rdotfill\ #2}}

\def\chapter#1{\newpage
		 \advance\kapnum by 1
		 \secnum=0
		 \leftline{\bmf \the\kapnum. #1}
		 \vskip 5pt
		 \writetoc\tkap{#1}\par	
               }

\def\section#1{\medskip\nobreak
		\advance\secnum by 1
		\leftline{\fsec \the\kapnum.\the\secnum. #1}\nobreak
		\writetoc\tsec{#1}\par\nobreak}
				
% odsazený odstavec (něco jako description v LaTeXu)
\def\hangpar #1\cr#2\par{{\parindent=0pt % dvojité {{ jsou tu proto, aby
                         \hangafter=1    % tyto parametry neovlivnily
                         \hangindent=30pt % ostatní odstavce mimo \hangpar
                         {\sc #1}\hfil\break
                         #2\par}}
% kolečkovaný item
\def\itbul{\item{$\bullet$}}

%%%%%%%%%%%%%%%%%%%%%%%%%
%%%%%%%% 
%%%%%%%%%%%%%%%%%%%%%%%%%



%%%%%%%%%%%%%%%%%%%%%%%%%
%%%%%%%% Titulní stránka
%%%%%%%%%%%%%%%%%%%%%%%%%
\nopagenumbers
\hbox{ }
\vfill
\leftline{\ftitul Můj počítač}
\vskip 2pt
\hrule depth 5pt
\vskip 2pt
\rightline{\fautor Henryk Paluch}
\newpage
%%%%%%%%%%%%%%%%%%%%%%%%%
%%%%%%%% Obsah
%%%%%%%%%%%%%%%%%%%%%%%%%
\footline={\hss\tenrm\folio\hss}
\pageno=-1
\heretoc           
\maketoc           
\newpage
\nopagenumbers
\baselineskip=12pt      % obsah nám trošku změnil řádkování
\lineskip=1pt           % tak to zase dáme do pořádku...
\lineskiplimit=0pt
%%%%%%%%%%%%%%%%%%%%%%%%%
%%%%%%%%% Úvod 
%%%%%%%%%%%%%%%%%%%%%%%%%
{\fuvnap \noindent Úvodem}
\vskip 5pt
{\fuvtext
Hlavním důvodem proč píši tuto knihu, je vyznat se v dnešní džungli hardware.
Zažil jsem spoustu nezapomenutelných chvílí při konfigurování svého počítače
a~nerad bych to prožíval znovu. Proto jsem se rozhodl stvořit tento malý
průvodce. Pokud pomůže i někomu jinému, než jsem já, budu jen potěšen.
Nuže, obraťte list\dots

}
%%%%%%%%%%%%%%%%%%%%%%%%%
\newpage
\pageno=1
\footline={\hss\tenrm\folio\hss}
%%%%%%%%%%%%%%%%%%%%%%%%%
%%%%%%%%% Kapitola I. základní popis
%%%%%%%%%%%%%%%%%%%%%%%%%
\def\oldsym{\dag}
\def\newsym{$\star$}
\def\krat{$\times$}
\chapter{Základní popis}
V této kapitole najdete podrobné údaje o zařízeních, která používám.
Symbolem \newsym\ jsem označil v současné době používaný hardware a
 symbolem \oldsym\ původní (obvykle již zastaralý) hardware. %fuj to jsou hsnusné věty

\section{Původní komponenty '93}
\item{\oldsym}
   Motherboard {\sc 386SXG2A Turbo AT 386SX 25/33 MHz}, CPU {\sc AMD
   386SX/SXL~33}, chipset {\sc Headland HT18/C WK41017}, frekvence 33/8 MHz,
   původně 2MB, později 4MB 32pin SIMM
\item{\oldsym}
   I/O karta neznámého původu (Čína?), čip {\sc TS8460}, 1\krat IDE,
    1\krat FDC , 2\krat S
   (16450), 1\krat P, 1\krat Game
\item{\oldsym}
   SVGA karta {\sc REALTEK RTG3105, 9230C}, {\sc CTC 104776AE-80},
   512kB, max. 1024\krat 768 neprokládané,  VESA BIOS 
\item{\newsym}
   Floppy mechanika $5^{1/4}$ {\sc Panasonic JU-475-4 A27}
\item{\newsym}
   Floppy mechanika $3^{1/2}$ {\sc YE-DATA YD-702 B-6037B B}
\item{\newsym}
   Pevný disk IDE {\sc IBM: WDA-L42, 40MB w/32KB Cache, CHS=977/5/17,
   MaxMult=64}
   
\section{Rok '94}
\item{\newsym}
   Pevný disk IDE {\sc QUANTUM LPS340A, 325MB w/98KB Cache, CHS=1011/15/44,
   MaxMult=8}
\section{Rok '95}
\item{\newsym}
   zvuková karta SF16-VIP, ViBRA-16, 16bitů 44kHz stereo, 2\krat 4W, 4\krat CD-ROM
   rozhraní na IDE, Panasonic, Mitsumi a Sony

\section{Nové komponenty '96}
\item{\newsym}
   základní deska {\sc Octek HIPPO 10 486}, CPU AMD486DX4/100, 16M 72pin
   SIMM, integrovaná I/O karta, 2\krat IDE, 1\krat FDC, 2\krat S, 1\krat P,
   1\krat Game
\item{\newsym}
   grafická karta {\sc VL-VGA-28}, chipset {\sc Cirrus Logic CLDG-5428},
   1MB RAM
\item{\newsym}
   IDE CD-ROM {\sc  F HO P.1 0, ATAPI, CDROM drive}, BTC Behavior Group,
   6\krat speed, ISO9660 (High Sierra), CD-ROM/XA, Multisession, 
   Photo-CD, CD+

\section{Poznatky z provozu}
\hangpar MB386SX\cr
 výborná kompatibilita s různými operačními systémy (LINUX, FreeBSD, NetBSD,
 a dokonce i OS/2 (je tam zmínka i v originální příručce k MB --- v roce 93),
 což je takřka neuvěřitelné. S původním FDC řadičem však
 má novější LINUX (1.1.40? a vyšší) problémy --- DMA přetečení. Stačí však
 vyhodit nadbytečná volání {\tt floppy\_request\_irq\_and\_dma()} například tak,
 že se ve
 {\tt floppy\_release\_irq\_and\_dma()} tvrdě napíše {\tt usage\_count=2; } a
 náhle vše funguje. Neznám žádný jiný počítač s tímto problémem\dots
 
\hangpar SVGA Realtek\cr
 s určitými programy se ráda kouše (The Lost Vikings, Unreal Demo), pomůže
 zpomalení pomocí Windows (to jsi rád Bille, co?). Snad
 jde o příliš rychlý přístup na paletové registry, ale nikdy se mi to
 nepodařilo crašnout vlastním programem.
 Přestože je tato karta
 zcela nestandardní a neumí nic víc než základní SVGA rozlišení, má jednu
 neuvěřitelnou věc --- VESA BIOS(!) --- na rok 93 neuvěřitelné\dots
 
\hangpar zvuková karta\cr
 je problematické zařízení. Největší problém způsobuje Plug \& Play (jak se
 dalo čekat). Co je nejhorší --- po zapnutí počítače se karta vždy zresetuje
 do implicitní konfigurace a je nutné spustit dosácký program, aby se
 překonfigurovala. Na druhé straně, díky originálnímu čipu (?) od Creative
 Technology, je ViBRA skutečně 100\% kompatibilní - dokonce i drivery pro
 OS/2 ji najdou (ve kterých je důsledná proti-klonová kontrola --- specialita
 firmy Creative Labs!). Při určité konstelaci příkazů a hardware ta potvůrka
 ráda vynechává sem tam nějaký bajtík pří rec/play (přesněji - nejdříve
 vynechá 1, pak 2\dots). Může to někdo potvrdit?

\hangpar hardisky\cr
 QUANTUM je překvapivě rychlé, ale již po necelých dvou letech se objevilo
 pár vadných sektorů, naštěstí nepřibývají další. Pokud je IBM zapojené jako
 Slave, pak se objeví zajímavý problém --- občasné zasekávání disků, údajně
 kvůli ztraceným přerušením (myslím, že přerušení se neztratí, ale řadič
 Master disku není ochoten příliš dlouho čekat). Řešení zpočívá v použití
 multisektorových operací (LINUX - příkaz {\tt hdparm -m xx /dev/hdb}),
 NetBSD 1.1 jej používá automaticky, OS/2 Warp pravděpodobně taky. Pod DOSem
 se díky jeho neefektivitě nic neprojeví. Kvalita IBM hardisku je
 neuvěřitalná --- při 2MB RAM jsem na něm nekolikrát kompiloval LINUXí jádro
 (asi 20 hodin nepřetržitého swapování) a dosud na něm není jediný vadný
 sektor

\hangpar MB Octek\cr
 zatím jsem zjistil nekompatibilitu řadiče klávesnice. Systémy pracující v
chráněném režimu (LINUX, Warp) nejsou schopné zresetovat počítač. OS/2 Warp
potřebuje novější {\tt IBMKBD.SYS} jinak se preventivně rovnou kousne. Další
problém zpočívá v integrovaném I/O řadiči. Je nutné programově povolit
sekundární řadič, jumpery nestačí. Ačkoliv se dodávají dokonce i ovladače
pro OS/2, ty zas neznají IDE CD-ROM, takže na
optimalizace je nejlepší se vykašlat. Firemní ovladače umí využít 32bitový
přístup na IDE periferie (neplést s tim paskvilem ve Windows ---~řadič
umí provádět transormaci 16~--~32 bitů) a je dokonce schopen zakázat zápis
do MBR. Umí také velice efektivně regulovat přenosovou rychlost. Jen je
hloupé, že to všechno potřebuje zvláštní ovladač!

 
\section{Přehled přerušení}

Zjistit přiřazeni jednotlivých přerušení je v počítačích IBM PC
kompatibilních přímo nadlidský úkol. Kdykoliv objevím někde seznam vektorů
přerušení, vždy se líší přinejmenším v úplnosti. Proto i já přícházím se
svojí troškou do mlýna.

\bigskip
\centerline{
\vbox{\offinterlineskip
      \hrule
      \halign{&\vrule#&\strut\quad#\hfil\quad\cr
        height 2pt&\omit&&\omit&\cr
        &\hfil Číslo přerušení&&\hfil Popis&\cr
        height 2pt&\omit&&\omit&\cr
        \noalign{\hrule}
        \noalign{\hrule}
        &IRQ0&&časovač&\cr
        &IRQ1&&klávesnice&\cr
        &IRQ2&&přesměrováno na IRQ9 (?)&\cr
        &IRQ3&&2. seriový port&\cr
        &IRQ4&&1. seriový port&\cr
        &IRQ5&&2. paralelní port&\cr
        &IRQ6&&FDC&\cr
        &IRQ7&&1. paralelní port&\cr
        \noalign{\hrule}
        &IRQ8&&přerušení hodin RTC&\cr
        &IRQ9&&nejčastěji přerušení od VGA&\cr
        &IRQ10&& ??? &\cr
        &IRQ11&& ??? &\cr
        &IRQ12&& ??? &\cr
        &IRQ13&& kooprocesor &\cr
        &IRQ14&& 1. IDE řadič &\cr
        &IRQ15&& 2. IDE řadič &\cr
        height 2pt&\omit&&\omit&\cr
        }
      \hrule
      }  
  }    
\bigskip

\itbul
  IRQ8-15 jsou dostupná pouze pro 16bitové karty

\itbul
  přerušení RTC je implicitně zakázáno, abyste jej využili, musíte
  odblokovat RTC

\itbul
  pokud nějaké přerušení používá dodatečně instalovaný hardware, bývá vhodné
  je nejdříve odmaskovat (porty {\tt 21H}, {\tt A1H}).

\itbul
  pokud potvrzujete IRQ8--15, pak musíte vyslat \uv{outy} oběma řadičům!!!
  Tedy:\hfil\break
  {\tt \obeylines
  mov al,20H
  out 0A0H,al
  out 020H,al
  \hbox{ }
  }
  
\itbul 
  protože INT 08H--0FH koliduje s vyjímkami v chráněném režimu, musí  DOS
 extendery a také všechny pořádné OS přeprogramovat řadič přerušení. Např.
 extender {\tt go32.exe} je přesune do INT 80H--88H. Není bez zajímavosti,
 že službě na nastavení oblsuhy přerušení v chráněném režimu zadáváte číslo
 {\bf nepřesměrovaného} vektoru (např. 08H), ale obsluze pro realný režim
 musíte zadat 80H !!!. Dodávám, že to, která rutina se vyvolá, závisí na tom,
 zda v době, kdy nastalo přerušení, je procesor zrovna v realném nebo
 chráněném režimu (!) Tyto vlastnosti byly pozorovány pod extendrem {\tt
 go32.exe} bez DPMI.

\section{DMA řadič}

Nejmystičtější komponenta PC. Originální DMA řadič od Intelu byl určen pro
8-bitové počítače, proto by vás nemělo překvapit, že délka přenosu je 
64kB pro kanály 0--3 a 128kB pro 4--7. Musíte navíc zajistit, aby nedošlo
k překročení {\bf fyzické} hranice 64kB. Nyní už asi chápete, proč DMA
potřebuje navíc Page Registry. Mimochodem, 8-bitové kanály jsou obvykle
schopné adresovat jen 1MB, i když někdy i 16MB. 


\bigskip
\centerline{
\vbox{\offinterlineskip
      \hrule
      \halign{&\vrule#&\strut\quad#\hfil\quad\cr
        height 2pt&\omit&&\omit&\cr
        &\hfil Číslo kanálu&&\hfil Popis&\cr
        height 2pt&\omit&&\omit&\cr
        \noalign{\hrule depth 2pt}
        &0&&XT --- DRAM refresh, AT --- volný&\cr
        &1&&IBM SDLC (=volný)&\cr
        &2&&Floppy&\cr
        &3&&1. IDE řadič&\cr
        \noalign{\hrule}
        &4&&kaskáda&\cr
        &5&&volný ?&\cr
        &6&&volný ? &\cr
        &7&&volný ?&\cr
        height 2pt&\omit&&\omit&\cr
        }
      \hrule
      }  
  }    
\bigskip


\itbul
  IDE harddisky obvykle nepřenášejí data pomocí DMA (ve standardu ATA je
  implementace DMA {\bf nepovinná}. Některá dema (Mečiarnátor) však
  DMA přenos používají
  
\itbul
  jedinnou spolehlivou programátorskou příručkou k DMA řadiči jsou zdrojáky
  LINUXu ({\tt asm/dma.h}). Vyplatí se nevěřit sebevědomým
  knihám. Dalším užitečným zdojem je {\tt sblaster.zip}.
  

%%%%%%%%%%%%%%%%%%%%%%%%%
%%%%%%%%% Kapitola II. konektory
%%%%%%%%%%%%%%%%%%%%%%%%%
\chapter{Konektory na základní desce}
\section{Napájení}

\def\rbbox#1{\vbox to 10pt{\vfil\hbox{\eightrm #1}\vfil}}
\def\konnap#1{\hbox{%
             \hbox{\consym 3}
             \vbox{\baselineskip=10pt
                   \lineskiplimit=-\maxdimen
                   \rbbox{1#1}
                   \rbbox{2#1}
                   \rbbox{3#1}
                   \rbbox{4#1}
                   \rbbox{5#1}
                   \rbbox{6#1}
                   }
                  }
            }      

\medskip
\hbox{\hskip 20pt
\hbox{ \valign{\vfil#\vfil\cr
        \hbox to 40pt{\vbox{%
              \baselineskip=60pt
              \lineskiplimit=-\maxdimen
              \konnap{a}
              \konnap{b}
             }\hfil}\cr
\hbox{\qquad
\vbox{\offinterlineskip
      \hrule
      \halign{&\vrule#& \strut\quad#\hfil\quad\cr
        height2pt&\omit&&\omit&&\omit&\cr
        &\hfil PIN&&\hfil Barva&&\hfil Popis&\cr
        height2pt&\omit&&\omit&&\omit&\cr
        \noalign{\hrule depth 2pt}
        &1a&&???&&Power Good&\cr
        &2a&&???&&+5V&\cr
        &3a&&???&&+12V&\cr
        &4a&&???&&-12V&\cr
        &5a&&černá&&GND&\cr
        &6a&&černá&&GND&\cr
        \noalign{\hrule}
        &1b&&černá&&GND&\cr
        &2b&&černá&&GND&\cr
        &3b&&???&&-5V&\cr
        &4b&&???&&+5V&\cr
        &5b&&???&&+5V&\cr
        &6b&&???&&+5V&\cr
        }
      \hrule
      }}\cr
        }
     } 
   }   
\medskip        
      
\section{Turbo přepínač}

\medskip
\hbox{\hskip 20pt
  \valign{\vfil#\vfil\cr
    \hbox to 40pt{\vbox{\offinterlineskip
      \baselineskip=10pt
      \lineskiplimit=-\maxdimen
      \hbox{{\consym 1}\ \raise 2.5pt\hbox{\eightrm 1}}      
      \hbox{{\consym 1}\ \raise 2.5pt\hbox{\eightrm 2}}      
         }\hfil}\cr
    \hbox{\qquad
      \vbox{\offinterlineskip
       \hrule
       \halign{&\vrule#& \strut\quad#\hfil\quad\cr
         height2pt&\omit&&\omit&&\omit&\cr
         &\hfil PIN&&\hfil Barva&&\hfil Popis&\cr
         height2pt&\omit&&\omit&&\omit&\cr
         \noalign{\hrule depth 2pt}
         &1&&bílá&&signál&\cr
         &2&&žlutá&&zem&\cr
         &3&&černá&&signál &\cr
        }
      \hrule
      }}\cr
     }
   }      
\medskip         

\itbul
 Ačkoliv Turbo přepínač má 3 vodiče, na MB jsou jen 2 kontakty. Je možné si
 vybrat, jestli bude turbo režim při zapnutém nebo vypnutém přepínači???

\section{Turbo LED}
 
\medskip
\hbox{\hskip 20pt
  \valign{\vfil#\vfil\cr
    \hbox to 40pt{\vbox{\offinterlineskip
      \baselineskip=10pt
      \lineskiplimit=-\maxdimen
      \hbox{{\consym 1}\ \raise 2.5pt\hbox{\eightrm 1}}      
      \hbox{{\consym 1}\ \raise 2.5pt\hbox{\eightrm 2}}      
         }\hfil}\cr
    \hbox{\qquad
      \vbox{\offinterlineskip
       \hrule
       \halign{&\vrule#& \strut\quad#\hfil\quad\cr
         height2pt&\omit&&\omit&&\omit&\cr
         &\hfil PIN&&\hfil Barva&&\hfil Popis&\cr
         height2pt&\omit&&\omit&&\omit&\cr
         \noalign{\hrule depth 2pt}
         &1&&šedá&&LED-&\cr
         &2&&červená/nic&&+5V&\cr
        }
      \hrule
      }}\cr
     }
   }      
\medskip         

\itbul
  LED dioda se chová jako normální dioda s tím rozdílem, že v propustném
směru svítí (protékající proud ale nesmí být příliš velký, obvykle asi
20~mA, jinak si budete muset pořídit novou).
Pokud jí nahodou zapojíte obráceně, pak sice nebude svítit, ale nic
se nestane


\section{Reset}
 
\medskip
\hbox{\hskip 20pt
  \valign{\vfil#\vfil\cr
    \hbox to 40pt{\vbox{\offinterlineskip
      \baselineskip=10pt
      \lineskiplimit=-\maxdimen
      \hbox{{\consym 1}\ \raise 2.5pt\hbox{\eightrm 1}}      
      \hbox{{\consym 1}\ \raise 2.5pt\hbox{\eightrm 2}}      
         }\hfil}\cr
    \hbox{\qquad
      \vbox{\offinterlineskip
       \hrule
       \halign{&\vrule#& \strut\quad#\hfil\quad\cr
         height2pt&\omit&&\omit&&\omit&\cr
         &\hfil PIN&&\hfil Barva&&\hfil Popis&\cr
         height2pt&\omit&&\omit&&\omit&\cr
         \noalign{\hrule depth 2pt}
         &1&&bílá&&signal&\cr
         &2&&červená&&GND&\cr
        }
      \hrule
      }}\cr
     }
   }      
\medskip         

\itbul
  vzhledem k tomu, že Reset je obyčejný spínač, nemusíte si s polaritou
lámat hlavu

\section{Speaker}

\medskip
\hbox{\hskip 20pt
  \valign{\vfil#\vfil\cr
    \hbox to 40pt{\vbox{\offinterlineskip
      \baselineskip=10pt
      \lineskiplimit=-\maxdimen
      \hbox{{\consym 1}\ \raise 2.5pt\hbox{\eightrm 1}}      
      \hbox{{\consym 1}\ \raise 2.5pt\hbox{\eightrm 2}}      
      \hbox{{\consym 1}\ \raise 2.5pt\hbox{\eightrm 3}}      
      \hbox{{\consym 1}\ \raise 2.5pt\hbox{\eightrm 4}}      
         }\hfil}\cr
    \hbox{\qquad
      \vbox{\offinterlineskip
       \hrule
       \halign{&\vrule#& \strut\quad#\hfil\quad\cr
         height2pt&\omit&&\omit&&\omit&\cr
         &\hfil PIN&&\hfil Barva&&\hfil Popis&\cr
         height2pt&\omit&&\omit&&\omit&\cr
         \noalign{\hrule depth 2pt}
         &1&&černá&&Data out&\cr
         &2&&nic&&nezapojeno&\cr
         &3&&nic&&GND&\cr
         &4&&červená&&+5V&\cr 
        }
      \hrule
      }}\cr
     }
   }      
\medskip         

\itbul
  dokud nehodláte připojit výstup od PC Speakeru na Sound Blaster, pak se
  rovněž nemusíte zabývat polaritou
  
\section{Power LED \& KeyLock}

\medskip
\hbox{\hskip 20pt
  \valign{\vfil#\vfil\cr
    \hbox to 40pt{\vbox{\offinterlineskip
      \baselineskip=10pt
      \lineskiplimit=-\maxdimen
      \hbox{{\consym 1}\ \raise 2.5pt\hbox{\eightrm 1}}      
      \hbox{{\consym 2}\ \raise 2.5pt\hbox{\eightrm 2}}      
      \hbox{{\consym 1}\ \raise 2.5pt\hbox{\eightrm 3}}      
      \hbox{{\consym 1}\ \raise 2.5pt\hbox{\eightrm 4}}      
      \hbox{{\consym 1}\ \raise 2.5pt\hbox{\eightrm 5}}      
         }\hfil}\cr
    \hbox{\qquad
      \vbox{\offinterlineskip
       \hrule
       \halign{&\vrule#& \strut\quad#\hfil\quad\cr
         height2pt&\omit&&\omit&&\omit&\cr
         &\hfil PIN&&\hfil Barva&&\hfil Popis&\cr
         height2pt&\omit&&\omit&&\omit&\cr
         \noalign{\hrule depth 2pt}
         &1&&zelená&&LED+&\cr
         &2&&nic&&klíč&\cr
         &3&&černá&&GND (LED-)&\cr
         &4&&modrá&&KeyLock&\cr
         &5&&černá&&GND&\cr 
        }
      \hrule
      }}\cr
     }
   }      
\medskip         
  
  
      
\chapter{Konektory na periferiích}
\section{I/O konektory}
 
\section{Disketové mechaniky}  

\section{Hard disky}

\section{CD ROM}

sal sdlfkajslf ajsklfaslf jslkaf jaslk fasf
as fljsa fl aslfk aslfk jaslkf kaslfjsalfjkaslf 

\bye
