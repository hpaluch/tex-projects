%
% TeX News, (C) 1996 Henryk Paluch
% Formát: csPlain-TeX
% Kódování: UTF-8
% Použivejte jen a pouze na sve vlastni riziko!
%
%
\chyph % requires csplain
\nopagenumbers
% embedded picture (as PK font) is in this resolution
\pdfpkresolution=300

\hsize=190mm
%\hoffset=10mm
\font\svabach=eufb10 at 2.5cm
\font\velky=csr10 at 2.5cm
\font\tensv=eufb10
\font\eightrm=csr8
\font\fiverm=csr5
\font\ninerm=csr9
\font\bigtex=csr12 scaled \magstep4
\font\twelf=csbx12
\font\fourt=csbx10 scaled \magstep2
\font\fourtsym=cmsy10 scaled \magstep2
\font\maxif=csbx12 scaled \magstep3
\font\poznf=csssbx10
\font\wordf=cmbxti10
% Zde zacinaji TeX News
{\parindent=0pt
\hrule                  % vrchní čára
\vskip 1.5pt            % mezera
\hbox to \hsize{%       % titulek + "copyright"
   \vrule
   \hfil
    \hbox{{\velky \strut{\TeX\ }}{\svabach NEWS}}
    \hfil\vrule\hfil
   \vbox{\vfil
   \vbox{\hsize=2cm%
         \hrule height 2pt width 0pt
         \centerline{\TeX\ {\tensv NEWS}}
         \vskip 3pt
         {\fiverm
         \def\hpx#1{\centerline{#1: H.~Paluch}\vskip -5pt}
         \hpx{Připravil}
         \hpx{Napsal}
         \hpx{Vysázel}
         \hpx{Vylepšil}
         \hpx{Vytiskl}
          }
         }
   \vfil}
   \hfil
    \vrule%
     }
\vskip 1.5pt
\hrule height 1pt
\centerline{\fourt      % 14 bodovy font
        \vrule height 15pt depth 5pt width 0pt  % mezera nad a pod  datem..
        \def\mujbul{\raise.1ex\hbox{$\bullet$}\ }
        \textfont2=\fourtsym  % 14 bodovy font pro $\bullet$ !!!
        \mujbul PRO VŠECHNY \TeX NIKY \mujbul 4.~června 1996\ \mujbul
        }
\hrule height 1pt
\vskip 1.5pt
% nyní přijdou hlavní 3 boxy
%
\newdimen\dimenone
\newdimen\dimentwo
\newdimen\dimenthree
\newdimen\hlavnivyska
\dimenone=0.20000\hsize
\dimentwo=0.50000\hsize
\dimenthree=0.20000\hsize
\hlavnivyska=19.5cm
\newbox\boxone
\newbox\boxtwo
\newbox\boxthree
\setbox\boxthree=\vbox{\hsize=\dimenone
                \eightrm \hbadness=1000 \hfuzz=.1pt
                \emergencystretch=2em
                {\vrule height 14.5pt width 0pt depth 6pt \tenrm
                \twelf Proč jsme tady?}
                {\it
                \parskip=1pt plus 5pt minus .1pt
                Hlavním popudem vydat tyto noviny byl jistý
                J.~K.~z O. Tento občan neustále napadal geniální
                \TeX~a neoprávněně vychvaloval jakýsi Word,
                ačkoliv každému je přece jasné, že \TeX\ se svou
                jedinečnou kvalitou nedá vůbec srovnávat
                s~\uv{jedním slovem}.

                \parindent=5pt
                A proto jsem se rozhodl, že těm hloupým Wordistům
                dám co proto!

                Stručně řečeno \TeX\ kritizují obvykle jen ti, co ho
                neznají, resp. ho neznají příliš dokonale. Jak se dá
                jinak vysvětlit, že si někteří myslí, že {\sl něco }
                není v \TeX u možné!

                \TeX\ se vyznačuje následujícími schopnostmi:
                \def\mujit#1\par{\parindent=10pt
                        \item{--} #1 \par}

                \mujit skutečná nezávislost výstupu na zařízení. \TeX\
                       generuje DVI (Device Independent), který je 100\%
                       nezávislý na výstupním zařízení (pokud úmyslně
                       nepoužijeme {\tt $\backslash$special})


                \mujit Pozice znaků jsou uloženy s větší přesnosti než je vlnová
                       délka světla, tj. optickými prostředky jsou
                       odchylky neměřitelné

                \mujit kvalitní algoritmus formátování odstavce. Je
                       možné detailně nastavit všechny tolerance při lámáni
                       odstavce. Dělení slov (až 256 jazyků současně) je
                       samozřejmostí

                \mujit dynamické formátování dokumentu. Je možné přesně
                specifikovat doporučenou, maximální a minimální
                přípustnou mezeru mezi slovy, řádky a odstavci (tzv.
                glue - lepidlo). Díky
                tomu se může \TeX\ snadno přizpůsobit požadavkům sazeče
                (např. automatické vyplnění stránky)

                \parindent=5pt
                Myslím si, že možnosti \TeX u hovoří samy za sebe\dots

                \vskip 2.5pt
                \hrule
                {\noindent
                {\vrule height 14.5pt width 0pt depth 6pt%
                \twelf Obrázky}
                \par}

                \noindent
                Mnozí nezna\TeX ové si myslí, že \TeX\ není schopen
                pracovat s obrázky. Skutečnost pochopitelně není tak
                černá. Existuje hned několik možností:
                \def\mujb#1\par{\parindent=10pt
                        \item{$\bullet$} #1 \par}
                \font\mflogo=logo10
                \mujb pomocí {\mflogo METAFONT}u. Vektorové obrázky
                      můžeme vytvářet rovnou zde, přičemž máme k
                      dispozici mj. nejen goniometrické funkce, ale také
                      Bézierovy křivky (co na to {\wordf Word}?)
                \mujb na import bitmap, GIF, PCX apod. můžeme použít
                      konverzní program {\tt bm2font}
                \mujb můžeme využít konverze do PostScriptu a
                      naimportovat si EPS

                  }
                }
\setbox\boxtwo=\vbox{\hsize=\dimentwo%
                    \leftline{\maxif Přichází nová \TeX nologie}
                    \vskip -2pt
                    \leftline{\poznf Jak \TeX\ změnil svět}
                    \parindent=5pt
                    Je obdivuhodné, že tak velkolepé dílo jako \TeX\
                    napsal jediný člověk --- Donald Ervin Knuth. Jako by
                    to bylo málo, autor se navíc rozhodl, že celou svojí
                    práci dá k dispozici všem ostatním --- \TeX\ je
                    možné zdarma získat (např. na internetu).

                    Jedním z poznávacích znaků \TeX u je vlastní rodina
                    fontů zvaná {\sl Computer Modern}. Tato písma se
                    vyznačují estetickým vzhledem a dobrou čitelností.
                    Naproti tomu, ke komerčním systémum se vždy
                    (pochopitelně) dodávají fonty, které jsou
                    nejlevnější, ale také nepříliš pohledné. Typickými
                    zástupci jsou {\sl Helvetica} a {\sl Times New
                    Roman}.


                    \moveright.50\dimentwo\hbox to 0pt{\vtop{\ninerm
                    \newdimen\varunit
                    \varunit=0.009\dimentwo
                    \baselineskip10.5pt plus 0pt
                    \parshape 9
                        -30.74\varunit 61.48\varunit
                        -44.19\varunit 88.39\varunit
                        -51.70\varunit 103.40\varunit
                        -55.72\varunit 111.45\varunit
                        -57.00\varunit 114.00\varunit
                        -55.72\varunit 111.45\varunit
                        -51.70\varunit 103.40\varunit
                        -44.19\varunit 88.39\varunit
                        -30.74\varunit 61.48\varunit
                        \noindent
                        \hbadness 6000
                        \tolerance 9999
                        \pretolerance 0
                        Je to k neuvěření, ale řada lidí se mylně
                        domnívá, že \TeX\ není schopen flexibilně
                        formátovat odstavec do předem zadaného tvaru.
                        Jak je patrné jde o pouhou domněnku, která není
                        příliš na místě. Je možné zadat odsazení a délku
                        každého řádku v odstavci -- hlavní výhodou je,
                        že \TeX\ sám zajistí správné lámaní textu do
                        odstavce. Není tedy problém naformátovat si text
                        dle libosti a vkusu. Zajímalo by mě, milí
                        Wordisté, jak by si s něčím takovým poradil váš
                        dosud oblíbený \uv{Wordprocesor}.

                        }\hss%
                      }
                      \vskip 1.5pt
                      \hrule
                      \vskip 1.5pt
                      \leftline{\maxif Tabulky}
                      Tvorba tabulek v \TeX u nemusí být vždy triviální
                      záležitostí, na druhou stranu jsem obdařeni velkou
                      flexibilitou, která je omezena jen našimi
                      možnostmi. Mezi další výhody patří pro \TeX\
                      typická absolutní přesnost (není třeba se obávat
                      Wordovské tolerance \uv{plus mínus autobus}).

                      \def\LaTeX{L\kern-.36em\raise.5ex\hbox{\sevenrm
                                                      A}\kern-.12em\TeX}
                      \def\AMS{$\cal A\kern-.166em\lower.5ex\hbox{$\cal
                                                     M$}\kern-.075em S$}


  %                      \vskip 10pt
                \vglue 10pt plus 50pt minus 0pt

                      \centerline{%
                      \vbox{\offinterlineskip
                            \hrule         % uvodni linka
                            \halign{&\vrule#&\quad\hfil#\quad\cr
                            height2pt&\omit&&\omit&&\omit&\cr % mala mezirka
                            &\strut Program\hfil&&Verze\hfil&&Cena\hfil&\cr    % hlavicka
                             height2pt&\omit&&\omit&&\omit&\cr % mezera s carama
                             \noalign{\hrule height 2pt}
                             \noalign{\vskip -2pt}
                             height2pt&\omit&&\omit&&\omit&\cr % mezera s carama
                             &\strut{} &&plain-\TeX\hfil&&0,- Kč&\cr
                             &{}&&\multispan3\hrulefill&\cr
                             &\strut{} &&\AMS-\TeX \hfil&&0,- Kč&\cr
                             &{}&&\multispan3\hrulefill&\cr
 &\vbox to 0pt{\vss\hbox{\hss\hbox{\bigtex \TeX}\hss}\vskip 1pt}\hfil%
                                  &&\strut \LaTeX\hfil&&0,- Kč&\cr
                              \noalign{\hrule}
                              \noalign{\vskip -1pt}
                             height3pt&\omit&&\omit&&\omit&\cr % mezera s carama
                             \noalign{\vskip -1pt}
                             \noalign{\hrule}
                             &\strut{}{\wordf Word}\hfil&& 2.0,6.0,7.0\hfil&&%
                               $\infty$,- Kč&\cr
                              }
                            \hrule
                            }
                         }

                \vglue 18pt plus 100pt minus 0pt
                \hrule
                \vskip 1.5pt
                \leftline{\maxif Přenositelnost}
                \vskip -2pt
                {\poznf \noindent
                Přenositelnost programů začíná být důležitým aspektem
                při jejich vývoji.\par
                }
                \vskip 4pt
                % Pozor! nyní definujeme trojici boxů
                \newbox\pboxone
                \newbox\pboxtwo
                \newbox\pboxthree
                \setbox\pboxthree=\vbox{\tolerance=9999 \hsize=0.295\dimentwo
                \emergencystretch=1em
                Je logické, že autor \TeX u věnoval této otázce mnoho
                pozornosti. Díky tomu existují implementace \TeX u pro
                UNIX (včetně {\bf LINUX}u!), OS/2, AMIGA OS, Atari ST, Mac,
                VMS, DOS a dokonce i Windows (pro sado-maso).
                }
                \splittopskip=0pt
                \setbox\pboxone=\vsplit\pboxthree to 82pt
                \setbox\pboxtwo=\vbox to 88pt{\hsize=0.295\dimentwo
                          \font\linux=linuxa at 72.27truept
                          \vfil
                          \centerline{\linux\hbox{\char0\char1}}
                          \vskip 1.5pt
                          \centerline{\it Nejlepší OS}
                          \vfil}
                \vskip 1.5pt
                \line{\vbox to 88pt{\box\pboxone\vfil}\hfil%
                       \vrule\hfil\box\pboxtwo\hfil\vrule%
                       \hfil\vbox to 88pt{\box\pboxthree\vfil}\hfil}
                }
\setbox\boxone=\vsplit\boxthree to \hlavnivyska
\line{\vrule\hfil\box\boxone\hfil%
      \vrule\hfil\vbox to \hlavnivyska{\box\boxtwo\vfil}\hfil%
      \vrule\hfil\vbox to \hlavnivyska{\box\boxthree}\hfil\vrule}
\vskip 1.5pt
\hrule
}
\bye
